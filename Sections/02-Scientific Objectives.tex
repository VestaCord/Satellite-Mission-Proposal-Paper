\section{Science Objectives}
The Tonle Sap river's annual monsoon flow reversal causes the lake to expand to five times its dry season size and move enormous amounts of sediment, organic material, fish eggs, and young fish \citep{Regan2021} which are responsible for supporting the entire fishing industry and ecology. Satellite data can provide in real-time the current statistics, for comparison with historical trends.

A dedicated LEO nanosatellite earth observer will fill temporal gaps in aerospatial data on the water level, flow, trash, sediment levels in the Mekong region, which will explain changes in volume of fishing and biodiversity in the Mekong region, so as to inform responses to drought and a catastrophic fishing economy and ecology. This will clarify trends still being established by other earth observers, and include sensors calibrated specifically for water body observation. A focus on fluvial geomorphology will also aid policymakers to identify sources of the degradation of the annual monsoon flow reversal.

Satellites also help cover large areas inaccessible by land, provide a whole view of the entire river basin, and offer consistent and long-term monitoring to detect gradual changes and trends with diverse equipment such as LiDAR, SAR, and Hyperspectrometers. However, on-the-ground efforts are still paramount for a hollistic understanding.

\subsection{Literature Review}
Local governments, Mekong River Commission (MRC) and grassroots efforts are responsible for managing the development of the Mekong River. However, China and MRC have been obstructionist to qualifying and managing the impact of hydroelectric installations on droughts, despite that damming is known to disrupt thermal and flow regulation of other large river bodies. \citep{Yang2022}

From space, grassroots work has been supported by NASA's open-science SERVIR-Mekong, Earth Observation Fleet, demonstrating enhanced Mekong flood response, drought Resilience, crop yield security, and reduced forest fires \citep{SERVIR-Mekong}.

In other water bodies, water availability prediction models for conservation have been successfully constructed using just Google Earth and regression AI \citep{Evans2021}, yet SERVIR has neglected such areas, with satellite data missing on the the death of biodiversity, health of fisheries, reduced clean water availability, and fluvial geomorphology of the Mekong.

Thus, there is a data niche to be filled by a cross selection of data with NASA's MODIS and IKONOS, and a new LEO orbiting nanosatellite with RAAN matching daytime over the Mekong. This possibility was addressed in a 2007 satellite-driven report of the Mekong\citep{Gupta2007}, but it is more necessary now, 20 years later, as the monsoons are failing to bring 70-90\% of fish into Tonle Sap.