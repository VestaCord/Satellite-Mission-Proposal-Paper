% Introduction
\section{Introduction}
The Mekong River, one of the world's most significant waterways, serves as a lifeline for millions of people across Southeast Asia. However, the construction of dams along the Mekong has introduced a myriad of maritime and sustainability issues, destabilizing the region around Singapore. This proposal aims to qualify the impacts of this ecological disaster, by using satellites SAR and employing altimetry to study water flow and quality. Uniquely, satellite data available can provide information of high veracity to better advise policymakers and scientists in Southeast Asia, supporting Singapore's interests in a stable, secure, and sustainable region.


% Contribution
\subsection{Contribution}

\begin{itemize}
    \item Replicate that the rapid expansion of hydropower is more important to the decimation of lower Mekong fisheries than extra low rainfall and the El Nino effect.
	\item Investigating changes in water levels and flow patterns in local basins, lakes, and rivers, and impacts on  droughts and water scarcity.
	\item Assess levels of trash, sediment, and pollutants in the water, as well as the occurrence of algal blooms during monsoon pulses, to understand broader environmental degradation.
	\item Predict effects on fishing yields, crop production, biodiversity, of the Indochina region.
\end{itemize}

%     Straight replications: studies that verify whether a finding that has been already published can be repeated.  Replication and extension: similar to the one above, but with an adjustment.   Extension of a new theory/method in a new area.    Integrative review (e.g., meta-analysis).    New theories to explain an old phenomenon, possibly also including a comparison between an existing and the new theory against each other to find out which one works better.    Identifications of new phenomena worth of attention.   Grand syntheses that integrate earlier theories together.    New theories that predict new phenomena.
